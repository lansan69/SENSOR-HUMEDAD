% ======== Librerías utilizadas en el proyecto ========

\usepackage[utf8]{inputenc}        % Permite usar caracteres UTF-8 (acentos, eñes, etc.)
\usepackage[backend=biber]{biblatex} % Manejo de bibliografía con Biber
\usepackage[a4paper, 11pt,top=3cm, bottom=3cm, left=3cm, right=3cm]{geometry} % Configuración de márgenes y tamaño de papel
\usepackage[T1]{fontenc}           % Codificación de fuente para caracteres especiales
\usepackage{multirow}              % Permite combinar varias filas en tablas
\usepackage{booktabs}              % Mejora el diseño de tablas (líneas más elegantes)
\usepackage{graphicx}              % Inserción de imágenes
\usepackage{rotating}              % Permite rotar objetos (tablas, figuras, texto)
\usepackage{float}                 % Control preciso de la posición de figuras/tablas
\usepackage{fancyhdr}              % Personalización de encabezados y pies de página
\usepackage{setspace}              % Control del interlineado
\usepackage{listings}              % Inserción de código fuente con formato
\usepackage{ragged2e}              % Permite justificar texto a la izquierda, derecha o completo
\usepackage{enumitem}              % Control del formato de listas (viñetas, numeración, etc.)
\usepackage{hyperref}              % Enlaces y referencias internas/externas en el PDF
\usepackage{xcolor}                % Permite usar y definir colores
\usepackage{amsmath,amssymb,textcomp} % Símbolos y entornos matemáticos avanzados
\usepackage{multicol}              % Permite dividir texto en varias columnas
\usepackage{tikz}                  % Creación de gráficos vectoriales y diagramas
\usepackage{pdflscape}             % Permite páginas en orientación horizontal (landscape)
\usepackage{tabularx}              % Tablas con ancho ajustable automáticamente
\usepackage{longtable}             % Tablas que pueden extenderse a múltiples páginas

\usepackage[spanish]{babel}
\usepackage{colortbl} % Para usar colores en las tablas
\usepackage{xcolor} % Para definir colores
\usepackage[utf8]{inputenc}
\usepackage{newfloat}
\usepackage{caption}
\usepackage{rotating} % in your preamble
\usepackage{lipsum}

% Header & footer
\pagestyle{fancy}
\fancyhf{}
\setlength{\parindent}{0in}
